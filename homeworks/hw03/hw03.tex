\documentclass[12pt]{article}

%packages
%\usepackage{latexsym}
\usepackage{graphicx}
\usepackage{color}
\usepackage{amsmath}
\usepackage{dsfont}
\usepackage{placeins}
\usepackage{amssymb}
\usepackage{wasysym}
\usepackage{abstract}
\usepackage{hyperref}
\usepackage{etoolbox}
\usepackage{datetime}
\usepackage{xcolor}
\usepackage{alphalph}
\settimeformat{ampmtime}

%\usepackage{pstricks,pst-node,pst-tree}

%\usepackage{algpseudocode}
%\usepackage{amsthm}
%\usepackage{hyperref}
%\usepackage{mathrsfs}
%\usepackage{amsfonts}
%\usepackage{bbding}
%\usepackage{listings}
%\usepackage{appendix}
\usepackage[margin=1in]{geometry}
%\geometry{papersize={8.5in,11in},total={6.5in,9in}}
%\usepackage{cancel}
%\usepackage{algorithmic, algorithm}

\makeatletter
\def\maxwidth{ %
  \ifdim\Gin@nat@width>\linewidth
    \linewidth
  \else
    \Gin@nat@width
  \fi
}
\makeatother

\definecolor{fgcolor}{rgb}{0.345, 0.345, 0.345}
\newcommand{\hlnum}[1]{\textcolor[rgb]{0.686,0.059,0.569}{#1}}%
\newcommand{\hlstr}[1]{\textcolor[rgb]{0.192,0.494,0.8}{#1}}%
\newcommand{\hlcom}[1]{\textcolor[rgb]{0.678,0.584,0.686}{\textit{#1}}}%
\newcommand{\hlopt}[1]{\textcolor[rgb]{0,0,0}{#1}}%
\newcommand{\hlstd}[1]{\textcolor[rgb]{0.345,0.345,0.345}{#1}}%
\newcommand{\hlkwa}[1]{\textcolor[rgb]{0.161,0.373,0.58}{\textbf{#1}}}%
\newcommand{\hlkwb}[1]{\textcolor[rgb]{0.69,0.353,0.396}{#1}}%
\newcommand{\hlkwc}[1]{\textcolor[rgb]{0.333,0.667,0.333}{#1}}%
\newcommand{\hlkwd}[1]{\textcolor[rgb]{0.737,0.353,0.396}{\textbf{#1}}}%

\usepackage{framed}
\makeatletter
\newenvironment{kframe}{%
 \def\at@end@of@kframe{}%
 \ifinner\ifhmode%
  \def\at@end@of@kframe{\end{minipage}}%
  \begin{minipage}{\columnwidth}%
 \fi\fi%
 \def\FrameCommand##1{\hskip\@totalleftmargin \hskip-\fboxsep
 \colorbox{shadecolor}{##1}\hskip-\fboxsep
     % There is no \\@totalrightmargin, so:
     \hskip-\linewidth \hskip-\@totalleftmargin \hskip\columnwidth}%
 \MakeFramed {\advance\hsize-\width
   \@totalleftmargin\z@ \linewidth\hsize
   \@setminipage}}%
 {\par\unskip\endMakeFramed%
 \at@end@of@kframe}
\makeatother

\definecolor{shadecolor}{rgb}{.77, .77, .77}
\definecolor{messagecolor}{rgb}{0, 0, 0}
\definecolor{warningcolor}{rgb}{1, 0, 1}
\definecolor{errorcolor}{rgb}{1, 0, 0}
\newenvironment{knitrout}{}{} % an empty environment to be redefined in TeX

\usepackage{alltt}
\usepackage[T1]{fontenc}

\newcommand{\qu}[1]{``#1''}
\newcounter{probnum}
\setcounter{probnum}{1}

%create definition to allow local margin changes
\def\changemargin#1#2{\list{}{\rightmargin#2\leftmargin#1}\item[]}
\let\endchangemargin=\endlist 

%allow equations to span multiple pages
\allowdisplaybreaks

%define colors and color typesetting conveniences
\definecolor{gray}{rgb}{0.5,0.5,0.5}
\definecolor{black}{rgb}{0,0,0}
\definecolor{white}{rgb}{1,1,1}
\definecolor{blue}{rgb}{0.5,0.5,1}
\newcommand{\inblue}[1]{\color{blue}#1 \color{black}}
\definecolor{green}{rgb}{0.133,0.545,0.133}
\newcommand{\ingreen}[1]{\color{green}#1 \color{black}}
\definecolor{yellow}{rgb}{1,1,0}
\newcommand{\inyellow}[1]{\color{yellow}#1 \color{black}}
\definecolor{orange}{rgb}{0.9,0.649,0}
\newcommand{\inorange}[1]{\color{orange}#1 \color{black}}
\definecolor{red}{rgb}{1,0.133,0.133}
\newcommand{\inred}[1]{\color{red}#1 \color{black}}
\definecolor{purple}{rgb}{0.58,0,0.827}
\newcommand{\inpurple}[1]{\color{purple}#1 \color{black}}
\definecolor{backgcode}{rgb}{0.97,0.97,0.8}
\definecolor{Brown}{cmyk}{0,0.81,1,0.60}
\definecolor{OliveGreen}{cmyk}{0.64,0,0.95,0.40}
\definecolor{CadetBlue}{cmyk}{0.62,0.57,0.23,0}

%define new math operators
\DeclareMathOperator*{\argmax}{arg\,max~}
\DeclareMathOperator*{\argmin}{arg\,min~}
\DeclareMathOperator*{\argsup}{arg\,sup~}
\DeclareMathOperator*{\arginf}{arg\,inf~}
\DeclareMathOperator*{\convolution}{\text{\Huge{$\ast$}}}
\newcommand{\infconv}[2]{\convolution^\infty_{#1 = 1} #2}
%true functions

%%%% GENERAL SHORTCUTS

%shortcuts for pure typesetting conveniences
\newcommand{\bv}[1]{\boldsymbol{#1}}

%shortcuts for compound constants
\newcommand{\BetaDistrConst}{\dfrac{\Gamma(\alpha + \beta)}{\Gamma(\alpha)\Gamma(\beta)}}
\newcommand{\NormDistrConst}{\dfrac{1}{\sqrt{2\pi\sigma^2}}}

%shortcuts for conventional symbols
\newcommand{\tsq}{\tau^2}
\newcommand{\tsqh}{\hat{\tau}^2}
\newcommand{\sigsq}{\sigma^2}
\newcommand{\sigsqsq}{\parens{\sigma^2}^2}
\newcommand{\sigsqovern}{\dfrac{\sigsq}{n}}
\newcommand{\tausq}{\tau^2}
\newcommand{\tausqalpha}{\tau^2_\alpha}
\newcommand{\tausqbeta}{\tau^2_\beta}
\newcommand{\tausqsigma}{\tau^2_\sigma}
\newcommand{\betasq}{\beta^2}
\newcommand{\sigsqvec}{\bv{\sigma}^2}
\newcommand{\sigsqhat}{\hat{\sigma}^2}
\newcommand{\sigsqhatmlebayes}{\sigsqhat_{\text{Bayes, MLE}}}
\newcommand{\sigsqhatmle}[1]{\sigsqhat_{#1, \text{MLE}}}
\newcommand{\bSigma}{\bv{\Sigma}}
\newcommand{\bSigmainv}{\bSigma^{-1}}
\newcommand{\thetavec}{\bv{\theta}}
\newcommand{\thetahat}{\hat{\theta}}
\newcommand{\thetahatmle}{\hat{\theta}_{\mathrm{MLE}}}
\newcommand{\thetavechatmle}{\hat{\thetavec}_{\mathrm{MLE}}}
\newcommand{\muhat}{\hat{\mu}}
\newcommand{\musq}{\mu^2}
\newcommand{\muvec}{\bv{\mu}}
\newcommand{\muhatmle}{\muhat_{\text{MLE}}}
\newcommand{\lambdahat}{\hat{\lambda}}
\newcommand{\lambdahatmle}{\lambdahat_{\text{MLE}}}
\newcommand{\etavec}{\bv{\eta}}
\newcommand{\alphavec}{\bv{\alpha}}
\newcommand{\minimaxdec}{\delta^*_{\mathrm{mm}}}
\newcommand{\ybar}{\bar{y}}
\newcommand{\xbar}{\bar{x}}
\newcommand{\Xbar}{\bar{X}}
\newcommand{\phat}{\hat{p}}
\newcommand{\Phat}{\hat{P}}
\newcommand{\Zbar}{\bar{Z}}
\newcommand{\iid}{~{\buildrel iid \over \sim}~}
\newcommand{\inddist}{~{\buildrel ind \over \sim}~}
\newcommand{\approxdist}{~{\buildrel approx \over \sim}~}
\newcommand{\equalsindist}{~{\buildrel d \over =}~}
\newcommand{\loglik}[1]{\ell\parens{#1}}
\newcommand{\thetahatkminone}{\thetahat^{(k-1)}}
\newcommand{\thetahatkplusone}{\thetahat^{(k+1)}}
\newcommand{\thetahatk}{\thetahat^{(k)}}
\newcommand{\half}{\frac{1}{2}}
\newcommand{\third}{\frac{1}{3}}
\newcommand{\twothirds}{\frac{2}{3}}
\newcommand{\fourth}{\frac{1}{4}}
\newcommand{\fifth}{\frac{1}{5}}
\newcommand{\sixth}{\frac{1}{6}}

%shortcuts for vector and matrix notation
\newcommand{\A}{\bv{A}}
\newcommand{\At}{\A^T}
\newcommand{\Ainv}{\inverse{\A}}
\newcommand{\B}{\bv{B}}
\newcommand{\K}{\bv{K}}
\newcommand{\Kt}{\K^T}
\newcommand{\Kinv}{\inverse{K}}
\newcommand{\Kinvt}{(\Kinv)^T}
\newcommand{\M}{\bv{M}}
\newcommand{\Bt}{\B^T}
\newcommand{\Q}{\bv{Q}}
\newcommand{\Qt}{\Q^T}
\newcommand{\R}{\bv{R}}
\newcommand{\Rt}{\R^T}
\newcommand{\Z}{\bv{Z}}
\newcommand{\X}{\bv{X}}
\newcommand{\Xsub}{\X_{\text{(sub)}}}
\newcommand{\Xsubadj}{\X_{\text{(sub,adj)}}}
\newcommand{\I}{\bv{I}}
\newcommand{\Y}{\bv{Y}}
\newcommand{\sigsqI}{\sigsq\I}
\renewcommand{\P}{\bv{P}}
\newcommand{\Psub}{\P_{\text{(sub)}}}
\newcommand{\Pt}{\P^T}
\newcommand{\Pii}{P_{ii}}
\newcommand{\Pij}{P_{ij}}
\newcommand{\IminP}{(\I-\P)}
\newcommand{\Xt}{\bv{X}^T}
\newcommand{\XtX}{\Xt\X}
\newcommand{\XtXinv}{\parens{\Xt\X}^{-1}}
\newcommand{\XtXinvXt}{\XtXinv\Xt}
\newcommand{\XXtXinvXt}{\X\XtXinvXt}
\newcommand{\x}{\bv{x}}
\newcommand{\onevec}{\bv{1}}
\newcommand{\oneton}{1, \ldots, n}
\newcommand{\yoneton}{y_1, \ldots, y_n}
\newcommand{\yonetonorder}{y_{(1)}, \ldots, y_{(n)}}
\newcommand{\Yoneton}{Y_1, \ldots, Y_n}
\newcommand{\iinoneton}{i \in \braces{\oneton}}
\newcommand{\onetom}{1, \ldots, m}
\newcommand{\jinonetom}{j \in \braces{\onetom}}
\newcommand{\xoneton}{x_1, \ldots, x_n}
\newcommand{\Xoneton}{X_1, \ldots, X_n}
\newcommand{\xt}{\x^T}
\newcommand{\y}{\bv{y}}
\newcommand{\yt}{\y^T}
\renewcommand{\c}{\bv{c}}
\newcommand{\ct}{\c^T}
\newcommand{\tstar}{\bv{t}^*}
\renewcommand{\u}{\bv{u}}
\renewcommand{\v}{\bv{v}}
\renewcommand{\a}{\bv{a}}
\newcommand{\s}{\bv{s}}
\newcommand{\yadj}{\y_{\text{(adj)}}}
\newcommand{\xjadj}{\x_{j\text{(adj)}}}
\newcommand{\xjadjM}{\x_{j \perp M}}
\newcommand{\yhat}{\hat{\y}}
\newcommand{\yhatsub}{\yhat_{\text{(sub)}}}
\newcommand{\yhatstar}{\yhat^*}
\newcommand{\yhatstarnew}{\yhatstar_{\text{new}}}
\newcommand{\z}{\bv{z}}
\newcommand{\zt}{\z^T}
\newcommand{\bb}{\bv{b}}
\newcommand{\bbt}{\bb^T}
\newcommand{\bbeta}{\bv{\beta}}
\newcommand{\beps}{\bv{\epsilon}}
\newcommand{\bepst}{\beps^T}
\newcommand{\e}{\bv{e}}
\newcommand{\Mofy}{\M(\y)}
\newcommand{\KofAlpha}{K(\alpha)}
\newcommand{\ellset}{\mathcal{L}}
\newcommand{\oneminalph}{1-\alpha}
\newcommand{\SSE}{\text{SSE}}
\newcommand{\SSEsub}{\text{SSE}_{\text{(sub)}}}
\newcommand{\MSE}{\text{MSE}}
\newcommand{\RMSE}{\text{RMSE}}
\newcommand{\SSR}{\text{SSR}}
\newcommand{\SST}{\text{SST}}
\newcommand{\JSest}{\delta_{\text{JS}}(\x)}
\newcommand{\Bayesest}{\delta_{\text{Bayes}}(\x)}
\newcommand{\EmpBayesest}{\delta_{\text{EmpBayes}}(\x)}
\newcommand{\BLUPest}{\delta_{\text{BLUP}}}
\newcommand{\MLEest}[1]{\hat{#1}_{\text{MLE}}}

%shortcuts for Linear Algebra stuff (i.e. vectors and matrices)
\newcommand{\twovec}[2]{\bracks{\begin{array}{c} #1 \\ #2 \end{array}}}
\newcommand{\threevec}[3]{\bracks{\begin{array}{c} #1 \\ #2 \\ #3 \end{array}}}
\newcommand{\fivevec}[5]{\bracks{\begin{array}{c} #1 \\ #2 \\ #3 \\ #4 \\ #5 \end{array}}}
\newcommand{\twobytwomat}[4]{\bracks{\begin{array}{cc} #1 & #2 \\ #3 & #4 \end{array}}}
\newcommand{\threebytwomat}[6]{\bracks{\begin{array}{cc} #1 & #2 \\ #3 & #4 \\ #5 & #6 \end{array}}}

%shortcuts for conventional compound symbols
\newcommand{\thetainthetas}{\theta \in \Theta}
\newcommand{\reals}{\mathbb{R}}
\newcommand{\complexes}{\mathbb{C}}
\newcommand{\rationals}{\mathbb{Q}}
\newcommand{\integers}{\mathbb{Z}}
\newcommand{\naturals}{\mathbb{N}}
\newcommand{\forallninN}{~~\forall n \in \naturals}
\newcommand{\forallxinN}[1]{~~\forall #1 \in \reals}
\newcommand{\matrixdims}[2]{\in \reals^{\,#1 \times #2}}
\newcommand{\inRn}[1]{\in \reals^{\,#1}}
\newcommand{\mathimplies}{\quad\Rightarrow\quad}
\newcommand{\mathlogicequiv}{\quad\Leftrightarrow\quad}
\newcommand{\eqncomment}[1]{\quad \text{(#1)}}
\newcommand{\limitn}{\lim_{n \rightarrow \infty}}
\newcommand{\limitN}{\lim_{N \rightarrow \infty}}
\newcommand{\limitd}{\lim_{d \rightarrow \infty}}
\newcommand{\limitt}{\lim_{t \rightarrow \infty}}
\newcommand{\limitsupn}{\limsup_{n \rightarrow \infty}~}
\newcommand{\limitinfn}{\liminf_{n \rightarrow \infty}~}
\newcommand{\limitk}{\lim_{k \rightarrow \infty}}
\newcommand{\limsupn}{\limsup_{n \rightarrow \infty}}
\newcommand{\limsupk}{\limsup_{k \rightarrow \infty}}
\newcommand{\floor}[1]{\left\lfloor #1 \right\rfloor}
\newcommand{\ceil}[1]{\left\lceil #1 \right\rceil}

%shortcuts for environments
\newcommand{\beqn}{\vspace{-0.25cm}\begin{eqnarray*}}
\newcommand{\eeqn}{\end{eqnarray*}}
\newcommand{\bneqn}{\vspace{-0.25cm}\begin{eqnarray}}
\newcommand{\eneqn}{\end{eqnarray}}

%shortcuts for mini environments
\newcommand{\parens}[1]{\left(#1\right)}
\newcommand{\squared}[1]{\parens{#1}^2}
\newcommand{\tothepow}[2]{\parens{#1}^{#2}}
\newcommand{\prob}[1]{\mathbb{P}\parens{#1}}
\newcommand{\cprob}[2]{\prob{#1~|~#2}}
\newcommand{\littleo}[1]{o\parens{#1}}
\newcommand{\bigo}[1]{O\parens{#1}}
\newcommand{\Lp}[1]{\mathbb{L}^{#1}}
\renewcommand{\arcsin}[1]{\text{arcsin}\parens{#1}}
\newcommand{\prodonen}[2]{\bracks{\prod_{#1=1}^n #2}}
\newcommand{\mysum}[4]{\sum_{#1=#2}^{#3} #4}
\newcommand{\sumonen}[2]{\sum_{#1=1}^n #2}
\newcommand{\infsum}[2]{\sum_{#1=1}^\infty #2}
\newcommand{\infprod}[2]{\prod_{#1=1}^\infty #2}
\newcommand{\infunion}[2]{\bigcup_{#1=1}^\infty #2}
\newcommand{\infinter}[2]{\bigcap_{#1=1}^\infty #2}
\newcommand{\infintegral}[2]{\int^\infty_{-\infty} #2 ~\text{d}#1}
\newcommand{\supthetas}[1]{\sup_{\thetainthetas}\braces{#1}}
\newcommand{\bracks}[1]{\left[#1\right]}
\newcommand{\braces}[1]{\left\{#1\right\}}
\newcommand{\set}[1]{\left\{#1\right\}}
\newcommand{\abss}[1]{\left|#1\right|}
\newcommand{\norm}[1]{\left|\left|#1\right|\right|}
\newcommand{\normsq}[1]{\norm{#1}^2}
\newcommand{\inverse}[1]{\parens{#1}^{-1}}
\newcommand{\rowof}[2]{\parens{#1}_{#2\cdot}}

%shortcuts for functionals
\newcommand{\realcomp}[1]{\text{Re}\bracks{#1}}
\newcommand{\imagcomp}[1]{\text{Im}\bracks{#1}}
\newcommand{\range}[1]{\text{range}\bracks{#1}}
\newcommand{\colsp}[1]{\text{colsp}\bracks{#1}}
\newcommand{\rowsp}[1]{\text{rowsp}\bracks{#1}}
\newcommand{\tr}[1]{\text{tr}\bracks{#1}}
\newcommand{\rank}[1]{\text{rank}\bracks{#1}}
\newcommand{\proj}[2]{\text{Proj}_{#1}\bracks{#2}}
\newcommand{\projcolspX}[1]{\text{Proj}_{\colsp{\X}}\bracks{#1}}
\newcommand{\median}[1]{\text{median}\bracks{#1}}
\newcommand{\mean}[1]{\text{mean}\bracks{#1}}
\newcommand{\dime}[1]{\text{dim}\bracks{#1}}
\renewcommand{\det}[1]{\text{det}\bracks{#1}}
\newcommand{\expe}[1]{\mathbb{E}\bracks{#1}}
\newcommand{\expeabs}[1]{\expe{\abss{#1}}}
\newcommand{\expesub}[2]{\mathbb{E}_{#1}\bracks{#2}}
\newcommand{\indic}[1]{\mathds{1}_{#1}}
\newcommand{\var}[1]{\mathbb{V}\text{ar}\bracks{#1}}
\newcommand{\cov}[2]{\mathbb{C}\text{ov}\bracks{#1, #2}}
\newcommand{\corr}[2]{\text{Corr}\bracks{#1, #2}}
\newcommand{\se}[1]{\mathbb{S}\text{E}\bracks{#1}}
\newcommand{\seest}[1]{\hat{\mathbb{S}\text{E}}\bracks{#1}}
\newcommand{\bias}[1]{\text{Bias}\bracks{#1}}
\newcommand{\derivop}[2]{\dfrac{\text{d}}{\text{d} #1}\bracks{#2}}
\newcommand{\partialop}[2]{\dfrac{\partial}{\partial #1}\bracks{#2}}
\newcommand{\secpartialop}[2]{\dfrac{\partial^2}{\partial #1^2}\bracks{#2}}
\newcommand{\mixpartialop}[3]{\dfrac{\partial^2}{\partial #1 \partial #2}\bracks{#3}}

%shortcuts for functions
\renewcommand{\exp}[1]{\mathrm{exp}\parens{#1}}
\renewcommand{\cos}[1]{\text{cos}\parens{#1}}
\renewcommand{\sin}[1]{\text{sin}\parens{#1}}
\newcommand{\sign}[1]{\text{sign}\parens{#1}}
\newcommand{\are}[1]{\mathrm{ARE}\parens{#1}}
\newcommand{\natlog}[1]{\ln\parens{#1}}
\newcommand{\oneover}[1]{\frac{1}{#1}}
\newcommand{\overtwo}[1]{\frac{#1}{2}}
\newcommand{\overn}[1]{\frac{#1}{n}}
\newcommand{\oneoversqrt}[1]{\oneover{\sqrt{#1}}}
\newcommand{\sqd}[1]{\parens{#1}^2}
\newcommand{\loss}[1]{\ell\parens{\theta, #1}}
\newcommand{\losstwo}[2]{\ell\parens{#1, #2}}
\newcommand{\cf}{\phi(t)}

%English language specific shortcuts
\newcommand{\ie}{\textit{i.e.} }
\newcommand{\AKA}{\textit{AKA} }
\renewcommand{\iff}{\textit{iff}}
\newcommand{\eg}{\textit{e.g.} }
\newcommand{\st}{\textit{s.t.} }
\newcommand{\wrt}{\textit{w.r.t.} }
\newcommand{\mathst}{~~\text{\st}~~}
\newcommand{\mathand}{~~\text{and}~~}
\newcommand{\ala}{\textit{a la} }
\newcommand{\ppp}{posterior predictive p-value}
\newcommand{\dd}{dataset-to-dataset}

%shortcuts for distribution titles
\newcommand{\logistic}[2]{\mathrm{Logistic}\parens{#1,\,#2}}
\newcommand{\bernoulli}[1]{\mathrm{Bernoulli}\parens{#1}}
\newcommand{\betanot}[2]{\mathrm{Beta}\parens{#1,\,#2}}
\newcommand{\stdbetanot}{\betanot{\alpha}{\beta}}
\newcommand{\multnormnot}[3]{\mathcal{N}_{#1}\parens{#2,\,#3}}
\newcommand{\normnot}[2]{\mathcal{N}\parens{#1,\,#2}}
\newcommand{\classicnormnot}{\normnot{\mu}{\sigsq}}
\newcommand{\stdnormnot}{\normnot{0}{1}}
\newcommand{\uniformdiscrete}[1]{\mathrm{Uniform}\parens{\braces{#1}}}
\newcommand{\uniform}[2]{\mathrm{U}\parens{#1,\,#2}}
\newcommand{\stduniform}{\uniform{0}{1}}
\newcommand{\geometric}[1]{\mathrm{Geometric}\parens{#1}}
\newcommand{\hypergeometric}[3]{\mathrm{Hypergeometric}\parens{#1,\,#2,\,#3}}
\newcommand{\exponential}[1]{\mathrm{Exp}\parens{#1}}
\newcommand{\gammadist}[2]{\mathrm{Gamma}\parens{#1, #2}}
\newcommand{\poisson}[1]{\mathrm{Poisson}\parens{#1}}
\newcommand{\binomial}[2]{\mathrm{Binomial}\parens{#1,\,#2}}
\newcommand{\negbin}[2]{\mathrm{NegBin}\parens{#1,\,#2}}
\newcommand{\rayleigh}[1]{\mathrm{Rayleigh}\parens{#1}}
\newcommand{\multinomial}[2]{\mathrm{Multinomial}\parens{#1,\,#2}}
\newcommand{\gammanot}[2]{\mathrm{Gamma}\parens{#1,\,#2}}
\newcommand{\cauchynot}[2]{\text{Cauchy}\parens{#1,\,#2}}
\newcommand{\invchisqnot}[1]{\text{Inv}\chisq{#1}}
\newcommand{\invscaledchisqnot}[2]{\text{ScaledInv}\ncchisq{#1}{#2}}
\newcommand{\invgammanot}[2]{\text{InvGamma}\parens{#1,\,#2}}
\newcommand{\chisq}[1]{\chi^2_{#1}}
\newcommand{\ncchisq}[2]{\chi^2_{#1}\parens{#2}}
\newcommand{\ncF}[3]{F_{#1,#2}\parens{#3}}

%shortcuts for PDF's of common distributions
\newcommand{\logisticpdf}[3]{\oneover{#3}\dfrac{\exp{-\dfrac{#1 - #2}{#3}}}{\parens{1+\exp{-\dfrac{#1 - #2}{#3}}}^2}}
\newcommand{\betapdf}[3]{\dfrac{\Gamma(#2 + #3)}{\Gamma(#2)\Gamma(#3)}#1^{#2-1} (1-#1)^{#3-1}}
\newcommand{\normpdf}[3]{\frac{1}{\sqrt{2\pi#3}}\exp{-\frac{1}{2#3}(#1 - #2)^2}}
\newcommand{\normpdfvarone}[2]{\dfrac{1}{\sqrt{2\pi}}e^{-\half(#1 - #2)^2}}
\newcommand{\chisqpdf}[2]{\dfrac{1}{2^{#2/2}\Gamma(#2/2)}\; {#1}^{#2/2-1} e^{-#1/2}}
\newcommand{\invchisqpdf}[2]{\dfrac{2^{-\overtwo{#1}}}{\Gamma(#2/2)}\,{#1}^{-\overtwo{#2}-1}  e^{-\oneover{2 #1}}}
\newcommand{\exponentialpdf}[2]{#2\exp{-#2#1}}
\newcommand{\poissonpdf}[2]{\dfrac{e^{-#1} #1^{#2}}{#2!}}
\newcommand{\binomialpdf}[3]{\binom{#2}{#1}#3^{#1}(1-#3)^{#2-#1}}
\newcommand{\rayleighpdf}[2]{\dfrac{#1}{#2^2}\exp{-\dfrac{#1^2}{2 #2^2}}}
\newcommand{\gammapdf}[3]{\dfrac{#3^#2}{\Gamma\parens{#2}}#1^{#2-1}\exp{-#3 #1}}
\newcommand{\cauchypdf}[3]{\oneover{\pi} \dfrac{#3}{\parens{#1-#2}^2 + #3^2}}
\newcommand{\Gammaf}[1]{\Gamma\parens{#1}}

%shortcuts for miscellaneous typesetting conveniences
\newcommand{\notesref}[1]{\marginpar{\color{gray}\tt #1\color{black}}}

%%%% DOMAIN-SPECIFIC SHORTCUTS

%Real analysis related shortcuts
\newcommand{\zeroonecl}{\bracks{0,1}}
\newcommand{\forallepsgrzero}{\forall \epsilon > 0~~}
\newcommand{\lessthaneps}{< \epsilon}
\newcommand{\fraccomp}[1]{\text{frac}\bracks{#1}}

%Bayesian related shortcuts
\newcommand{\yrep}{y^{\text{rep}}}
\newcommand{\yrepisq}{(\yrep_i)^2}
\newcommand{\yrepvec}{\bv{y}^{\text{rep}}}


%Probability shortcuts
\newcommand{\SigField}{\mathcal{F}}
\newcommand{\ProbMap}{\mathcal{P}}
\newcommand{\probtrinity}{\parens{\Omega, \SigField, \ProbMap}}
\newcommand{\convp}{~{\buildrel p \over \rightarrow}~}
\newcommand{\convLp}[1]{~{\buildrel \Lp{#1} \over \rightarrow}~}
\newcommand{\nconvp}{~{\buildrel p \over \nrightarrow}~}
\newcommand{\convae}{~{\buildrel a.e. \over \longrightarrow}~}
\newcommand{\convau}{~{\buildrel a.u. \over \longrightarrow}~}
\newcommand{\nconvau}{~{\buildrel a.u. \over \nrightarrow}~}
\newcommand{\nconvae}{~{\buildrel a.e. \over \nrightarrow}~}
\newcommand{\convd}{~{\buildrel \mathcal{D} \over \rightarrow}~}
\newcommand{\nconvd}{~{\buildrel \mathcal{D} \over \nrightarrow}~}
\newcommand{\withprob}{~~\text{w.p.}~~}
\newcommand{\io}{~~\text{i.o.}}

\newcommand{\Acl}{\bar{A}}
\newcommand{\ENcl}{\bar{E}_N}
\newcommand{\diam}[1]{\text{diam}\parens{#1}}

\newcommand{\taua}{\tau_a}

\newcommand{\myint}[4]{\int_{#2}^{#3} #4 \,\text{d}#1}
\newcommand{\laplacet}[1]{\mathscr{L}\bracks{#1}}
\newcommand{\laplaceinvt}[1]{\mathscr{L}^{-1}\bracks{#1}}
\renewcommand{\min}[1]{\text{min}\braces{#1}}
\renewcommand{\max}[1]{\text{max}\braces{#1}}

\newcommand{\Vbar}[1]{\bar{V}\parens{#1}}
\newcommand{\expnegrtau}{\exp{-r\tau}}

%%% problem typesetting
\definecolor{darkgrey}{rgb}{0.10,0.10,0.9}

\newcommand{\problem}[1]{\noindent \colorbox{black}{{\color{yellow} \large{\textsf{\textbf{Problem \arabic{probnum}}}}~}} \addtocounter{probnum}{1} \vspace{0.2cm} \\ \iftoggle{professormode}{}{\color{darkgrey}} #1}

\newcommand{\easysubproblem}[1]{\ingreen{\item} \iftoggle{professormode}{}{\color{darkgrey}} [easy] #1 \color{black} }
\newcommand{\intermediatesubproblem}[1]{\inorange{\item} \iftoggle{professormode}{}{\color{darkgrey}} [harder] #1 \color{black} }
\newcommand{\hardsubproblem}[1]{\inred{\item} \iftoggle{professormode}{}{\color{darkgrey}} [difficult] #1 \color{black} }
\newcommand{\extracreditsubproblem}[1]{\inpurple{\item} \iftoggle{professormode}{}{\color{darkgrey}} [E.C.] #1 \color{black} }


\newcommand{\spc}[1]{\iftoggle{professormode}{\\ \vspace{#1cm}}{\\ \vspace{-0.3cm}}}

\makeatletter
\newalphalph{\alphmult}[mult]{\@alph}{26}
\renewcommand{\labelenumi}{(\alphmult{\value{enumi}})}

\newcommand{\support}[1]{\text{Supp}\bracks{#1}}
\newcommand{\mode}[1]{\text{Mode}\bracks{#1}}
\newcommand{\IQR}[1]{\text{IQR}\bracks{#1}}
\newcommand{\quantile}[2]{\text{Quantile}\bracks{#1,\,#2}}


\newtoggle{professormode}
\toggletrue{professormode} %STUDENTS: DELETE or COMMENT this line



\title{MATH 341 / 650 Spring 2017 Homework \#3}

\author{Professor Adam Kapelner} %STUDENTS: write your name here

\iftoggle{professormode}{
\date{Due 2PM under my office door (KY604), Monday, March 6, 2017 \\ \vspace{0.5cm} \small (this document last updated \today ~at \currenttime)}
}

\renewcommand{\abstractname}{Instructions and Philosophy}

\begin{document}
\maketitle

\iftoggle{professormode}{
\begin{abstract}
The path to success in this class is to do many problems. Unlike other courses, exclusively doing reading(s) will not help. Coming to lecture is akin to watching workout videos; thinking about and solving problems on your own is the actual ``working out.''  Feel free to \qu{work out} with others; \textbf{I want you to work on this in groups.}

Reading is still \textit{required}. For this homework set, review the Binomial-Beta, then read about Bayesian Hypothesis Testing, Bayes Factors, Credible Regions and Empirical Bayes.

The problems below are color coded: \ingreen{green} problems are considered \textit{easy} and marked \qu{[easy]}; \inorange{yellow} problems are considered \textit{intermediate} and marked \qu{[harder]}, \inred{red} problems are considered \textit{difficult} and marked \qu{[difficult]} and \inpurple{purple} problems are extra credit. The \textit{easy} problems are intended to be ``giveaways'' if you went to class. Do as much as you can of the others; I expect you to at least attempt the \textit{difficult} problems.  Problems marked \qu{[MA]} are for the masters students only (those enrolled in the 650 course). For those in 341, doing these questions will count as extra credit.

This homework is worth 100 points but the point distribution will not be determined until after the due date. See syllabus for the policy on late homework.

Up to 10 points are given as a bonus if the homework is typed using \LaTeX. Links to instaling \LaTeX~and program for compiling \LaTeX~is found on the syllabus. You are encouraged to use \url{overleaf.com}. If you are handing in homework this way, read the comments in the code; there are two lines to comment out and you should replace my name with yours and write your section. The easiest way to use overleaf is to copy the raw text from hwxx.tex and preamble.tex into two new overleaf tex files with the same name. If you are asked to make drawings, you can take a picture of your handwritten drawing and insert them as figures or leave space using the \qu{$\backslash$vspace} command and draw them in after printing or attach them stapled.

The document is available with spaces for you to write your answers. If not using \LaTeX, print this document and write in your answers. I do not accept homeworks which are \textit{not} on this printout. Keep this first page printed for your records.

\end{abstract}

\thispagestyle{empty}
\vspace{1cm}
NAME: \line(1,0){380}
\clearpage
}

\problem{We will now be looking at the beta-prior, binomial-likelihood Bayesian model once again.}

\begin{enumerate}

\easysubproblem{Using the principle of indifference, what should the prior on $\theta$ (the parameter for the Bernoulli model) be?}\spc{1}

\easysubproblem{Let's say $n=6$ and your data is $0,1,0,1,0,1$. What is the likelihood of this event?}\spc{2}

\easysubproblem{Does it matter the order as to which the data came in? Yes/no.}\spc{0.5}

\intermediatesubproblem{Show that the unconditional joint probability (the denominator in Bayes rule) is a beta function and specify its two arguments. We did this in class.}\spc{3}

\intermediatesubproblem{Put your answer from (a), (b) and (d) together to find the posterior probability of $\theta$ given this dataset. Show that it is equal to a beta distribution and specify its parameters.}\spc{5}

\easysubproblem{Now imagine you are not indifferent and you have some idea about what $\theta$ could be a priori and that subjective feeling can be specified as a beta distribution. (1) Draw the basic shapes that the beta distribution can take on, (2) give an example of $\alpha$ and $\beta$ values that would produce these shapes and (3) write a sentence about what each one means for your prior belief. These shapes are in the notes.}\spc{20}

%\intermediatesubproblem{Imagine $n$ data points of which you don't know the realization values. Show that $\theta~|~X \sim \betanot{\alpha + x}{\beta + (n - x)}$. Note that $x := \sum_{i = 1}^n x_i$ which is the total number of successes and thereby $n-x$ is the total nubmer of failures. The answer is in the notes but try to do it yourself.}\spc{10}

\easysubproblem{What does it mean that the beta distribution is the \qu{conjugate prior} for the binomial likelihood?}\spc{3}

\intermediatesubproblem{Stare at that distribution, $\theta~|~X \sim \betanot{\alpha + x}{\beta + (n - x)}$. Some say the values of $\alpha$ and $\beta$ can be interpreted as follows: $\alpha - 1$ is considered the prior number of successes and $\beta - 1$ is considered the prior number of failures. Why is this a good interpretation? Writing out the PDF of $\theta~|~X$ should help you see it.}\spc{5}

\intermediatesubproblem{By the principle of indifference, how many successes and failures is that equivalent to seeing a priori?}\spc{3}

\easysubproblem{Why are large values of $\alpha$ and/or $\beta$ considered to compose a \qu{strong} prior?}\spc{2}

\intermediatesubproblem{[MA] What is the weakest prior you can think of and why?}\spc{5}

\hardsubproblem{[MA] Show that if $X \sim \stdbetanot$ then $\var{X} = \frac{\alpha\beta}{\squared{\alpha + \beta}(\alpha + \beta + 1)}$.}\spc{11}

\hardsubproblem{I think a priori that $\theta$ should be expected to be 0.8 with a standard error of 0.02. Solve for the values of $\alpha$ and $\beta$ based on my a priori specification. }\spc{9}

\hardsubproblem{Prove that the posterior predictive distribution is $X^*~|~X \sim \bernoulli{\frac{x + \alpha}{n + \alpha + \beta}}$. MA students --- do this yourself. Other students --- use my notes and justify each step. I use a property of the \href{https://en.wikipedia.org/wiki/Gamma_function}{gamma function}. Remember, if $W \sim \bernoulli{\theta}$ then $\prob{W=1} = \theta$. Use that trick! Set $X^* = 1$ and find that probability!}\spc{12}

\intermediatesubproblem{The frequentist estimate of $\theta$ is $\phat = 3/6 = 0.5$ So a frequentist would probably use a posterior predictive distribution (if he had such a thing) as $X^*~|~X \sim \bernoulli{0.5}$. Why conceptually does this answer differ from your answer in (n)?}\spc{7}

\easysubproblem{Assume the dataset in (b) where $n=6$. Assume $\theta \sim \betanot{\alpha=2}{\beta=2}$ a priori. Find the $\thetahatmap$, $\thetahatmmse$ and $\thetahatmae$ estimates for $\theta$.\\ 

For the $\thetahatmae$ estimate, you'll need to obtain a quantile of the beta distribution. Use \texttt{R} on your computer or online using \href{http://www.r-fiddle.org/}{R-Fiddle}. The \texttt{qbeta} function in \texttt{R} finds arbitrary beta quantiles. Its first argument is the quantile desired e.g. 2.5\%, the next is $\alpha$ and the third is $\beta$. So to find the 97.5\%ile of a $\betanot{\alpha=2}{\beta=2}$ for example you type \texttt{qbeta(.975, 2, 2)} into the \texttt{R} console.}\spc{5}

\intermediatesubproblem{Why are all three of these estimates the same?}\spc{5}

\easysubproblem{Write out an expression for the 95\% credible region for $\theta$. Then solve computationally using the \texttt{qbeta} function in \texttt{R}.}\spc{3}

\easysubproblem{Compute a 95\% frequentist CI for $\theta$.}\spc{3}

\hardsubproblem{Let $\mu : \reals \rightarrow \reals^+$ be the \href{https://en.wikipedia.org/wiki/Lebesgue_measure}{Lebesgue measure} which measures the length of a subset of $\reals$. Why is $\mu(\text{CR}) < \mu(\text{CI})$? That is, why is the Bayesian Confidence Interval tighter than the Frequentist Confidence Interval? Use your answers from (r) and (s). }\spc{3}

\easysubproblem{Explain the disadvantages of the highest density region method for computing credible regions.}\spc{3}

\end{enumerate}


\problem{We will again be looking at the beta-prior, binomial-likelihood Bayesian model and practicing hypothesis testing.}


\begin{enumerate}

\intermediatesubproblem{Design a prior where you believe $\expe{\theta} = 0.5$ and you feel as if your belief represents information contained in five coin flips.}\spc{3}

\intermediatesubproblem{Calculate a 95\% a priori credible region for $\theta$. Use \texttt{R} on your computer (or \href{http://www.r-fiddle.org/}{R-Fiddle} online) and its \texttt{qbeta} function.}\spc{3}

\easysubproblem{You flip the same coin 100 times and you observe 39 heads. Find the distribution of $\theta~|~X$.}\spc{0.5}


\easysubproblem{Calculate a 95\% a posteriori credible region for $\theta$.}\spc{1}

\easysubproblem{Why is your answer to (d) a smaller interval than (b)?}\spc{3}

\intermediatesubproblem{Test the hypothesis that this coin is fair given prior information from (a) and the data from (c). Use the credible region method. Make sure you say whether you retain or reject the null and justify why.}\spc{12}

\intermediatesubproblem{Test the hypothesis that this coin has a bias towards Heads (not tails) given prior information from (a) and the data from (c). Calculate the Bayesian $p$-val for the test to determine if you should retain or reject $H_0$.}\spc{6}

\easysubproblem{Let's say you wanted to test whether the coin is fair but you are indifferent to any $\theta$ which is different from 0.5 by a margin of 0.1. Write out the hypotheses for this test.}\spc{3}

\intermediatesubproblem{Test the hypotheses from (h) given prior information from (a) and the data from (c). Make sure you say whether you retain or reject the null and justify why.}\spc{10}


\intermediatesubproblem{Calculate the Bayesian $p$-val for the test in (i).}\spc{3}

\intermediatesubproblem{Given the tested hypotheses from (i) and data from (c), write the formula for the Bayes factor and then write the integral expression. Do not solve.}\spc{4}


\intermediatesubproblem{[MA] Calculate the Bayes Factor numerically by solving the integral expression in (k). Interpret your value of $K$ (or $B$) according to \href{https://en.wikipedia.org/wiki/Bayes_factor}{wikipedia page about Bayes Factors}.}\spc{11}

\easysubproblem{Given prior information from (a) and the data from (c), what is the distribution of one future coin flip?}\spc{1}

\intermediatesubproblem{Prove that the $\thetahatmmse$ is a shrinkage estimator.}\spc{8}

\easysubproblem{Assume again the prior information from (a). What is the shrinkage proportion $\rho$ for this prior when  estimating $\theta$ via $\thetahatmmse$?.}\spc{2}



\hardsubproblem{Prove that $\thetahatmmse$ is a biased estimator (i.e. its expectation is \textit{not} $\theta$).}\spc{5}

\easysubproblem{Prove that $\displaystyle \limitn \rho = 0$ and therefore this bias $\rightarrow 0$ as your dataset gets larger.}\spc{4}

\hardsubproblem{[MA] Why on Earth should anyone use shrinkage estimators if they're biased? Google it. Discuss.}\spc{6}

\end{enumerate}

\end{document}




\problem{This problem is concerned with the logical definition of probability. As a review, we have:

\begin{enumerate}[1.]
\item \textbf{The Objective View} --- This is the view that probabilities are properties of the physical world and can only be defined by either

\begin{enumerate}[(a)] 
\item its \textbf{Long Run Frequency} Seeing the same event over and over again and tabulating when the event occurs will create a frequency which will asympotically become the probability or
\item its \textbf{Propensity} which means deep down inside, the physical object is wired for events in certain proportions.
\end{enumerate}


Thus, events that are non physical such as the probability of Donald Trump winning the 2016 election is outside of the purview of probability. The objective view of probability is tied to the frequentist view of statistics. We also have the...

\item \textbf{The Epistemic View} --- This is the view that probabilities are inherently living inside the minds of human beings who are forced to grapple with uncertainty as they see it. Laplace believed probability is an illusion because we don't have certainty about the universe. The two definitions here are that probability...

\begin{enumerate}[(a)] 
\item is \textbf{Logical} which means that given the same information, everyone would come to the same conclusion.
\item is \textbf{Subjective} which means that given the same information, everyone would \textit{not} come to the same conclusion. Thus probability is defined as the degree of belief of some individual which differs from another individual.
\end{enumerate}

Thus events that are non physical such as the probability of Donald Trump winning the 2016 election (prior to the election) can now be legitimate probabilities as they can be computed.


\end{enumerate}
}

\begin{enumerate}

\easysubproblem{
We discussed last time that the logical definition requires the principle of indifference which goes by many different names. We will now go about showing that the principle of indifference has a tenuous foundation and thereby rendering the logical theory of probability inadequate. Thus, our conclusion will be that Bayesian Statistics runs on the Subjective definition which we may develop in a later homework. We begin with demonstrating a paradox in the logical definition for a discrete set of $\theta_0$. \\

Imagine you have a library with thousands of books but all are either red, green, yellow or purple but you don't know the proportions of the books' colors. Imagine you are blindfolded and select a random book and you are only interested if it's \textit{red} or \textit{not red}. According to the principle of indifference, what is your prior probability that the book is red? Remember, $|\Theta_0|  = 2$ here.}\spc{2}

\easysubproblem{Imagine you are blindfolded and select a random book and you are interested if it's red, green, yellow or purple. According to the principle of indifference, what is your prior probability that the book is red? Remember, $|\Theta_0|  = 4$ here.}\spc{1}


\intermediatesubproblem{Why do (a) and (b) constitute a paradox? Does this limit the application of the principle of indifference?}\spc{4}

\end{enumerate}

\end{document}

\problem{These are questions about McGrayne's book, preface, chapter 1, 2 and 3.}

\begin{enumerate}

\easysubproblem{Explain Hume's problem of induction with the sun rising every day.}\spc{3}

\easysubproblem{Explain the \qu{inverse probability problem.}}\spc{3}

\easysubproblem{What is Bayes' billiard table problem?}\spc{3}

\hardsubproblem{[MA] How did Price use Bayes' idea to prove the existence of the deity?} \spc{3}

\easysubproblem{Why should Bayes Rule really be called \qu{Laplace's Rule?}}\spc{3}

\hardsubproblem{Prove the version of Bayes Rule found on page 20. State your assumption(s) explicitly. Reference class notes as well.}\spc{4}

\easysubproblem{Give two scientific contexts where Laplace used inverse probability theory to solve major problems.}\spc{3}

\hardsubproblem{[MA] Why did Laplace turn into a frequentist later in life?} \spc{3}

\easysubproblem{State Laplace's version of Bayes Rule (p31).} \spc{3}

\easysubproblem{Why was Bayes Rule \qu{damned} (pp36-37)?} \spc{3}

\easysubproblem{According to Edward Molina, what is the prior (p41)?} \spc{3}

\easysubproblem{What is the source of the \qu{credibility} metric that insurance companies used in the 1920's?} \spc{3}

\easysubproblem{Can the principle of inverse probability work without priors? Yes/no} \spc{1}

\hardsubproblem{In class we discussed the \qu{principle of indifference} which is a term I borrowed from \href{http://www.amazon.com/Philosophical-Theories-Probability-Issues-Science/dp/041518276X/ref=sr_1_1?ie=UTF8&qid=1455112335&sr=8-1&keywords=donald+gillies+theory+of+probability}{Donald Gillies' Philosophical Theories of Probability}. On Wikipedia, it says that Jacob Bernoulli called it the \qu{principle of insufficient reason}. McGrayne in her research of original sources comes up with many names throughout history this principle was named. List all of them you can find here.} \spc{1}

\easysubproblem{Jeffreys seems to be the founding father of modern Bayesian Statistics. But why did the world turn frequentist in the 1920's? (p57)} \spc{1}
\end{enumerate}


\problem{More about likelihood estimators. These exercises are important because we will soon be doing normal mean estimation using the Bayesian shrinkage estimator.}

\begin{enumerate}

\easysubproblem{Write the PDF of $X \sim \normnot{\theta}{1^2}$.}\spc{3}

\hardsubproblem{Find the MLE for $\theta$ if $\Xoneton \iid \normnot{\theta}{1^2}$.}\spc{5}

\hardsubproblem{[MA] Find the MLE for $\theta$ if $\Xoneton \iid \normnot{\mu}{\sigsq}$. Solve the system of equations $\partialop{\mu}{\ell(\theta)} = 0$ and $\partialop{\sigsq}{\ell(\theta)} = 0$ where $\ell(\theta)$ denotes the log likelihood. You can easily find this online. But try to do it yourself.} \spc{17}

\end{enumerate}

\problem{This problem is concerned with one of the definitions of probability. In Math 241 I taught that there were two perspectives on probability which loosely yielded four definitions according to \href{http://www.amazon.com/Philosophical-Theories-Probability-Issues-Science/dp/041518276X/ref=sr_1_1?ie=UTF8&qid=1455112335&sr=8-1&keywords=donald+gillies+theory+of+probability}{Gillies}. As a review, we have:

\begin{enumerate}[1.]
\item \textbf{The Objective View} --- This is the view that probabilities are properties of the physical world and can only be defined by either

\begin{enumerate}[(a)] 
\item its \textbf{Long Run Frequency} Seeing the same event over and over again and tabulating when the event occurs will create a frequency which will asympotically become the probability or
\item its \textbf{Propensity} which means deep down inside, the physical object is wired for events in certain proportions
\end{enumerate}


Thus events that are non physical such as the probability of Donald Trump winning the 2016 election is outside of the purview of probability. The objective view of probability is tied to the frequentist view of statistics. We also have the...

\item \textbf{The Epistemic View} --- This is the view that probabilities are inherently living inside the minds of human beings who are forced to grapple with uncertainty as they see it. This was Laplace's view --- probability is an illusion because we don't have certainty about the universe. The two definitions here are that probability

\begin{enumerate}[(a)] 
\item is \textbf{Logical} which means that given the same information, everyone would come to the same conclusion.
\item is \textbf{Subjective} which means that given the same information, everyone would \textit{not} come to the same conclusion. Thus probability is defined as the degree of belief of some individual which differs from another individual.
\end{enumerate}

Thus events that are non physical such as the probability of Donald Trump winning the 2016 election can now be legitimate probabilities as they can be computed.\\

We will focus here on the logical definition.

\end{enumerate}
}

\begin{enumerate}

\easysubproblem{In the logical definition we see written \qu{given the same information}. What does \qu{information} mean here? $X$ or $\theta$?}\spc{1}

\easysubproblem{In the logical definition we see written \qu{would come to the same conclusion}. What does \qu{conclusion} mean here? Look at the four parts of the Bayes Rule equation for posterior inference and write the correct one.}\spc{1}

\easysubproblem{Given parts (a) and (b), what does everyone need to start with being the same for conclusions to be the same?}\spc{1}

\extracreditsubproblem{[MA] For some reason in order for the logical definition to work, Keynes required the principle of indifference which is a stronger assumption than everyone begins with equal priors. Why do you think that is?}\spc{4}


\easysubproblem{The logical definition requires employing the principle of indifference. State the principle of indifference for a discrete set $\Theta_0 = \braces{\theta_1, \ldots, \theta_m}$ by giving the prior distribution of $\theta$.}\spc{2}

\easysubproblem{State the principle of indifference for a continuous set $\Theta_0 = \bracks{a,~b}$ by giving the prior distribution of $\theta$.}\spc{2}

%\hardsubproblem{Show that if $\Theta_0 = \braces{\theta_1, \ldots}$ i.e. is countably infinite, the principle of indifference fails. Proof by contradiction is an easy strategy.}\spc{5}

%\hardsubproblem{[MA] Show that if $\Theta_0 = \reals$ or some subset of $\reals$ of infinite measure, the principle of indifference fails. Proof by contradiction is an easy strategy.}\spc{2}

\hardsubproblem{State a practical case where you would not want to use the principle of indifference. Think simple. Think coins and coin flips.}\spc{4} %put paradoxes on next homework...

\hardsubproblem{[MA] How is the objectivist view of probability related to the frequentist view of statistics?}\spc{4}

\hardsubproblem{[MA] How is the epistemic view of probability related to the Bayesian view of statistics?}\spc{10}

\end{enumerate}

\problem{\textbf{More coin flips.} The purpose of this exercise is for you to review the methods we did in class concerning the binomial likelihood and Bayesian inference for the parameter $\theta = p$.}

\begin{enumerate}


\easysubproblem{Assume that $\Theta_0 = \braces{0.1,~0.5}$. Using the principle of indifference, what is the prior?} \spc{1}

\intermediatesubproblem{Draw the space $\Theta_0 \times \mathcal{X}$ to scale just like we did in class. Assume you have three observations i.e. $ \mathcal{X} = \support{X}^3$.} \spc{10}

\intermediatesubproblem{Draw the space for $\Theta_0$ given the data was 0,0,0 to scale just like we did in class.} \spc{5}

\easysubproblem{Calculate $\cprob{\theta = 0.1}{X_1 = 0,~X_2 = 0,~X_3 = 0}$}. The picture above should help. \spc{3}

\easysubproblem{Assume for the rest of the problem that $\Theta_0 = \braces{\theta_1, \theta_2, \ldots, \theta_5}$. Assume the same data as we did in class $x_1 = 0,~x_2 = 1,~x_3 = 1$. Write out the likelihood. Your answer should not include any $\theta_i$'s but a general free variable $\theta$.}\spc{3}


\intermediatesubproblem{Right out an expression for the denominator using the sum notation. The denominator here would be $\prob{X_1 = 0,~X_2 = 1,~X_3 = 1}$.}\spc{3}

\intermediatesubproblem{Solve for the posterior probability of $\theta$ using your answers from the previous questions.} \spc{6}


\hardsubproblem{Write expressions for $\thetahatmap$, $\thetahatmae$ and $\thetahatmmse$, i.e. the three Bayesian point estimators we discussed in class.} \spc{5}

\hardsubproblem{Find the distribution $\cprob{X^*}{X_1 = 0,~X_2 = 1,~X_3 = 1}$ where $x^*$ is a new realization from the model $\mathcal{F}$ heretofore never seen. Sum signs are okay.} \spc{5}
\end{enumerate}



\problem{We will now be looking at the beta-prior, binomial-likelihood Bayesian model.}

\begin{enumerate}

\easysubproblem{Using the principle of indifference, what should the prior on $\theta$ (the parameter for the Bernoulli model) be?}\spc{1}

\easysubproblem{Let's say $n=6$ and your data is $0,1,0,1,0,1$. What is the likelihood of this event?}\spc{2}

\easysubproblem{Does it matter the order as to which the data came in? Yes/no.}\spc{0.5}

\intermediatesubproblem{Show that the unconditional joint probability (the denominator in Bayes rule) is a beta function and specify its two arguments. We did this in class.}\spc{5}

\intermediatesubproblem{Put your answer from (a), (b) and (d) together to find the posterior probability of $\theta$ given this dataset. Show that it is equal to a beta distribution and specify its parameters.}\spc{5}

\easysubproblem{Now imagine you are not indifferent and you have some idea about what $\theta$ could be a priori and that subjective feeling can be specified as a beta distribution. (1) Draw the five basic shapes that the beta distribution can take on, (2) give an example of $\alpha$ and $\beta$ values that would produce these shapes and (3) write a sentence about what each one means for your prior belief. These shapes are in the notes.}\spc{12}

\intermediatesubproblem{Imagine $n$ data points of which you don't know the realization values. Show that $\theta~|~X \sim \betanot{\alpha + x}{\beta + (n - x)}$. Note that $x := \sum_{i = 1}^n x_i$ which is the total number of successes and thereby $n-x$ is the total nubmer of failures. The answer is in the notes but try to do it yourself.}\spc{10}

\easysubproblem{What does it mean that the beta distribution is the \qu{conjugate prior} for the binomial likelihood?}\spc{3}

\extracreditsubproblem{Prove that $B(\alpha, \beta) = \frac{\Gammaf{\alpha} \Gammaf{\beta}}{\Gammaf{\alpha + \beta}}$.}
\end{enumerate}


\end{document}

\intermediatesubproblem{Stare at that distribution, $\theta~|~X \sim \betanot{\alpha + x}{\beta + (n - x)}$. Some say the values of $\alpha$ and $\beta$ can be interpreted as follows: $\alpha - 1$ is considered the prior number of successes and $\beta - 1$ is considered the prior number of failures. Why is this a good interpretation? Writing out the PDF of $\theta~|~X$ should help you see it.}\spc{5}

\intermediatesubproblem{By the principle of indifference, how many successes and failures is that equivalent to seeing a priori?}\spc{3}

\easysubproblem{Why are large values of $\alpha$ and/or $\beta$ considered to compose a \qu{strong} prior?}\spc{2}

\intermediatesubproblem{[MA] What is the weakest prior you can think of and why?}\spc{5}

\hardsubproblem{I think a priori that $\theta$ should be expected to be 0.8 with a standard error of 0.02. Solve for the values of $\alpha$ and $\beta$ based on my a priori specification. }\spc{7}

\hardsubproblem{Prove that the posterior predictive distribution is $X^*~|~X \sim \bernoulli{\frac{x + \alpha}{n + \alpha + \beta}}$. MA students --- do this yourself. Other students --- use my notes and justify each step. I use a property of the \href{https://en.wikipedia.org/wiki/Gamma_function}{gamma function}.}\spc{12}

\intermediatesubproblem{The frequentist estimate of $\theta$ is $\phat = 3/6 = 0.5$ So a frequentist would probably use a posterior predictive distribution (if he had such a thing) as $X^*~|~X \sim \bernoulli{0.5}$. Why conceptually does this answer differ from your answer in (n)?}\spc{4}

\easysubproblem{Assume the dataset in (b) where $n=6$. Assume $\theta \sim \betanot{\alpha=2}{\beta=2}$ a priori. Find the $\thetahatmap$, $\thetahatmmse$ and $\thetahatmae$ estimates for $\theta$.\\ 

For the $\thetahatmae$ estimate, you'll need to obtain a quantile of the beta distribution. Use \texttt{R} on your computer or online using \href{http://www.r-fiddle.org/}{R-Fiddle}. The \texttt{qbeta} function in \texttt{R} finds arbitrary beta quantiles. Its first argument is the quantile desired e.g. 2.5\%, the next is $\alpha$ and the third is $\beta$. So to find the 97.5\%ile of a $\betanot{\alpha=2}{\beta=2}$ for example you type \texttt{qbeta(.975, 2, 2)} into the \texttt{R} console.}\spc{5}

\intermediatesubproblem{Why are all three of these estimates the same?}\spc{5}

\easysubproblem{Write out an expression for the 95\% credible region for $\theta$. Then solve computationally using the \texttt{qbeta} function in \texttt{R}.}\spc{3}

\easysubproblem{Compute a 95\% frequentist CI for $\theta$.}\spc{3}

\hardsubproblem{Let $\mu : \reals \rightarrow \reals^+$ be the \href{https://en.wikipedia.org/wiki/Lebesgue_measure}{Lebesgue measure} which measures the length of a subset of $\reals$. Why is $\mu(\text{CR}) < \mu(\text{CI})$? That is, why is the Bayesian Confidence Interval tighter than the Frequentist Confidence Interval? Use your answers from (r) and (s). }\spc{3}

\easysubproblem{Explain the disadvantages of the highest density region method for computing credible regions.}\spc{3}

\end{enumerate}

\end{document}

\problem{These are questions about McGrayne's book, preface, chapter 1, 2 and 3.}

\begin{enumerate}

\easysubproblem{Explain Hume's problem of induction with the sun rising every day.}\spc{3}

\easysubproblem{Explain the \qu{inverse probability problem.}}\spc{3}

\easysubproblem{What is Bayes' billiard table problem?}\spc{3}

\hardsubproblem{[MA] How did Price use Bayes' idea to prove the existence of the deity?} \spc{3}

\easysubproblem{Why should Bayes Rule really be called \qu{Laplace's Rule?}}\spc{3}

\hardsubproblem{Prove the version of Bayes Rule found on page 20. State your assumption(s) explicitly. Reference class notes as well.}\spc{4}

\easysubproblem{Give two scientific contexts where Laplace used inverse probability theory to solve major problems.}\spc{3}

\hardsubproblem{[MA] Why did Laplace turn into a frequentist later in life?} \spc{3}

\easysubproblem{State Laplace's version of Bayes Rule (p31).} \spc{3}

\easysubproblem{Why was Bayes Rule \qu{damned} (pp36-37)?} \spc{3}

\easysubproblem{According to Edward Molina, what is the prior (p41)?} \spc{3}

\easysubproblem{What is the source of the \qu{credibility} metric that insurance companies used in the 1920's?} \spc{3}

\easysubproblem{Can the principle of inverse probability work without priors? Yes/no} \spc{1}

\hardsubproblem{In class we discussed the \qu{principle of indifference} which is a term I borrowed from \href{http://www.amazon.com/Philosophical-Theories-Probability-Issues-Science/dp/041518276X/ref=sr_1_1?ie=UTF8&qid=1455112335&sr=8-1&keywords=donald+gillies+theory+of+probability}{Donald Gillies' Philosophical Theories of Probability}. On Wikipedia, it says that Jacob Bernoulli called it the \qu{principle of insufficient reason}. McGrayne in her research of original sources comes up with many names throughout history this principle was named. List all of them you can find here.} \spc{1}

\easysubproblem{Jeffreys seems to be the founding father of modern Bayesian Statistics. But why did the world turn frequentist in the 1920's? (p57)} \spc{1}
\end{enumerate}


\problem{More about likelihood estimators. These exercises are important because we will soon be doing normal mean estimation using the Bayesian shrinkage estimator.}

\begin{enumerate}

\easysubproblem{Write the PDF of $X \sim \normnot{\theta}{1^2}$.}\spc{3}

\hardsubproblem{Find the MLE for $\theta$ if $\Xoneton \iid \normnot{\theta}{1^2}$.}\spc{3}

\hardsubproblem{[MA] Find the MLE for $\theta$ if $\Xoneton \iid \normnot{\mu}{\sigsq}$. Solve the system of equations $\partialop{\mu}{\ell(\theta)} = 0$ and $\partialop{\sigsq}{\ell(\theta)} = 0$ where $\ell(\theta)$ denotes the log likelihood. You can easily find this online. But try to do it yourself.} \spc{10}

\end{enumerate}

\problem{More about likelihood estimators. These exercises are important because we will soon be doing normal mean estimation using the Bayesian shrinkage estimator.}

\begin{enumerate}

\easysubproblem{Write the PDF of $X \sim \normnot{\theta}{1^2}$.}\spc{3}

\hardsubproblem{Find the MLE for $\theta$ if $\Xoneton \iid \normnot{\theta}{1^2}$.}\spc{5}

\hardsubproblem{[MA] Find the MLE for $\theta$ if $\Xoneton \iid \normnot{\mu}{\sigsq}$. Solve the system of equations $\partialop{\mu}{\ell(\theta)} = 0$ and $\partialop{\sigsq}{\ell(\theta)} = 0$ where $\ell(\theta)$ denotes the log likelihood. You can easily find this online. But try to do it yourself.} \spc{11}

\end{enumerate}

\problem{This problem is concerned with one of the definitions of probability. In Math 241 I taught that there were two perspectives on probability which loosely yielded four definitions according to \href{http://www.amazon.com/Philosophical-Theories-Probability-Issues-Science/dp/041518276X/ref=sr_1_1?ie=UTF8&qid=1455112335&sr=8-1&keywords=donald+gillies+theory+of+probability}{Gillies}. As a review, we have:

\begin{enumerate}[1.]
\item \textbf{The Objective View} --- This is the view that probabilities are properties of the physical world and can only be defined by either

\begin{enumerate}[(a)] 
\item its \textbf{Long Run Frequency} Seeing the same event over and over again and tabulating when the event occurs will create a frequency which will asympotically become the probability or
\item its \textbf{Propensity} which means deep down inside, the physical object is wired for events in certain proportions
\end{enumerate}


Thus events that are non physical such as the probability of Donald Trump winning the 2016 election is outside of the purview of probability. The objective view of probability is tied to the frequentist view of statistics. We also have the...

\item \textbf{The Epistemic View} --- This is the view that probabilities are inherently living inside the minds of human beings who are forced to grapple with uncertainty as they see it. This was Laplace's view --- probability is an illusion because we don't have certainty about the universe. The two definitions here are that probability

\begin{enumerate}[(a)] 
\item is \textbf{Logical} which means that given the same information, everyone would come to the same conclusion.
\item is \textbf{Subjective} which means that given the same information, everyone would \textit{not} come to the same conclusion. Thus probability is defined as the degree of belief of some individual which differs from another individual.
\end{enumerate}

Thus events that are non physical such as the probability of Donald Trump winning the 2016 election can now be legitimate probabilities as they can be computed.\\

We will focus here on the logical definition.

\end{enumerate}
}

\begin{enumerate}

\easysubproblem{In the logical definition we see written \qu{given the same information}. What does \qu{information} mean here? $X$ or $\theta$?}\spc{1}

\easysubproblem{In the logical definition we see written \qu{would come to the same conclusion}. What does \qu{conclusion} mean here? Look at the four parts of the Bayes Rule equation for posterior inference and write the correct one.}\spc{1}

\easysubproblem{Given parts (a) and (b), what does everyone need to start with being the same for conclusions to be the same?}\spc{1}

\extracreditsubproblem{[MA] For some reason in order for the logical definition to work, Keynes required the principle of indifference which is a stronger assumption than everyone begins with equal priors. Why do you think that is?}\spc{4}


\easysubproblem{The logical definition requires employing the principle of indifference. State the principle of indifference for a discrete set $\Theta_0 = \braces{\theta_1, \ldots, \theta_m}$ by giving the prior distribution of $\theta$.}\spc{2}

\easysubproblem{State the principle of indifference for a continuous set $\Theta_0 = \bracks{a,~b}$ by giving the prior distribution of $\theta$.}\spc{2}

\hardsubproblem{Show that if $\Theta_0 = \braces{\theta_1, \ldots}$ i.e. is countably infinite, the principle of indifference fails. Proof by contradiction is an easy strategy.}\spc{2}

\hardsubproblem{[MA] Show that if $\Theta_0 = \reals$ or some subset of $\reals$ of infinite measure, the principle of indifference fails. Proof by contradiction is an easy strategy.}\spc{2}

\hardsubproblem{State a practical case where you would not want to use the principle of indifference. Think simple. Think coins and coin flips.}\spc{4} %put paradoxes on next homework...

\hardsubproblem{[MA] How is the objectivist view of probability related to the frequentist view of statistics?}\spc{4}

\hardsubproblem{[MA] How is the epistemic view of probability related to the Bayesian view of statistics?}\spc{4}

\end{enumerate}

\problem{\textbf{More coin flips.} The purpose of this exercise is for you to review the methods we did in class concerning the binomial likelihood and Bayesian inference for the parameter $\theta = p$.}

\begin{enumerate}


\easysubproblem{Assume that $\Theta_0 = \braces{0.1,~0.5}$. Using the principle of indifference, what is the prior?} \spc{1}

\intermediatesubproblem{Draw the space $\Theta_0 \times \mathcal{X}$ to scale just like we did in class.} \spc{10}

\intermediatesubproblem{Draw the space for $\Theta_0$ given the data was 0,0,0 to scale just like we did in class.} \spc{5}

\easysubproblem{Calculate $\cprob{\theta = 0.1}{X_1 = 0,~X_2 = 0,~X_3 = 0}$}. The picture above should help. \spc{3}

\easysubproblem{Assume for the rest of the problem that $\Theta_0 = \braces{\theta_1, \theta_2, \ldots, \theta_5}$. Assume the same data as we did in class $x_1 = 0,~x_2 = 1,~x_3 = 1$. Write out the likelihood. Your answer should not include any $\theta_i$'s but a general free variable $\theta$.}\spc{3}


\intermediatesubproblem{Right out an expression for the denominator using the sum notation. The denominator here would be $\prob{X_1 = 0,~X_2 = 1,~X_3 = 1}$.}\spc{3}

\intermediatesubproblem{Solve for the posterior probability of $\theta$ using your answers from the previous questions.} \spc{4}


\intermediatesubproblem{Solve for the posterior probability of $\theta$ using your answers from the previous questions.} \spc{5}

\hardsubproblem{Write expressions for $\thetahatmap$, $\thetahatmae$ and $\thetahatmmse$, i.e. the three Bayesian point estimators we discussed in class.} \spc{5}

\hardsubproblem{Find the distribution $\cprob{X^*}{X_1 = 0,~X_2 = 1,~X_3 = 1}$ where $x^*$ is a new realization from the model $\mathcal{F}$ heretofore never seen. Sum signs are okay.} \spc{5}
\end{enumerate}


\problem{More about fundamentals of the Bayesian perspective.}

\begin{enumerate}


\intermediatesubproblem{Imagine in class we had a prior on $\theta$ in the Bernoulli family of indifferent between $\theta = 0.25$ and $\theta = 0.75$. What is the \textit{prior predictive distribution} i.e. what is the distribution of $X_1$ not seeing any data just given the prior. Draw the tree out as I did in class and it will be obvious. This is \textit{not} the posterior predictive distribution.} \spc{4}

\hardsubproblem{Give the formula in general for the prior predictive distribution for both the discrete and continuous case for general $\mathcal{F}$. Call it $\prob{X}$.} \spc{3}


\easysubproblem{Is this the same as the marginal likelihood / prior on the data as we've seen in class? Yes/no. Write a sentence giving more flesh to the definition of that pesky denominator $\prob{X}$.}\spc{2}


\intermediatesubproblem{More about the marginal likelihood... Explain why if the \qu{classic} idea of independence you learned in 241 i.e. $\prob{X} := \prob{\Xoneton} = \prod_{i=1}^n \prob{X_i}$ was applicable in the Bayesian perspective then no learning can be done from experience. Use the expression for the posterior predictive distribution to make a simple one-line statement.}\spc{3}

\intermediatesubproblem{[MA] We have showed that $\Xoneton$ are not independent. Show here that although they are not independent, they are identically distributed. You will need to use the property that $X_1~|~\theta, \ldots, X_n~|~\theta \iid p(x)$ which is known as \qu{conditional independence}. I know this is splitting hairs, but it's a good property to prove.}\spc{4}

\easysubproblem{[MA] We showed in class that $\Xoneton$ are not independent. Independence is in some sense \qu{too strong} of an assumption for Bayesian modeling. Instead, we use the weaker assumption called \qu{exchangeability}. Look in the book or on Google and give the definition for exchangeability.}\spc{3}


\easysubproblem{[MA] When we collect data for an experiment are the observations exchangeable? Is this an assumption we make use of all the time?}\spc{2}

\easysubproblem{[MA] Look up de Finetti's respresentation theorem. State it here.}\spc{6}

\easysubproblem{[MA] Does exchangeability imply conditional independence for some distribution $\prob{X~|~\theta}$? Yes / No.}\spc{3}
\end{enumerate}

\end{document}


%%%%%%%%%%%%%%%% HW 3


\problem{More about fundamentals of the Bayesian perspective.}

\begin{enumerate}


\hardsubproblem{Imagine in class we had a prior on $\theta$ (in the Bernoulli parametric family) that expressed indifference between $\theta = 0.25$ and $\theta = 0.75$. What is the \textit{prior predictive distribution} i.e. what is the distribution of $X_1$ not seeing any data just given the prior. Draw the tree out as I did in class and it will be obvious. This is \textit{not} the posterior predictive distribution.} \spc{4}

\intermediatesubproblem{Give the formula in general for the prior predictive distribution for both the discrete and continuous case for general $\mathcal{F}$. Call it $\prob{X}$.} \spc{3}


\easysubproblem{Is this the same as the \qu{marginal likelihood} or \qu{prior on the data} as we've seen in class? Yes/no.}\spc{2}


\hardsubproblem{More about the marginal likelihood... Explain why if the \qu{classic} idea of independence you learned in 241 i.e. $\prob{X} := \prob{\Xoneton} = \prod_{i=1}^n \prob{X_i}$ was applicable in the Bayesian perspective then no learning can be done from experience. Use the expression for the posterior predictive distribution to make a simple one-line statement.}\spc{3}

\intermediatesubproblem{[MA] We have showed that $\Xoneton$ are not independent. Show here that although they are not independent, they are identically distributed. You will need to use the property that $X_1~|~\theta, \ldots, X_n~|~\theta \iid p(x)$ which is known as \qu{conditional independence}. I know this is splitting hairs, but it's a good property to prove.}\spc{4}

\easysubproblem{[MA] We showed in class that $\Xoneton$ are not independent. Independence is in some sense \qu{too strong} of an assumption for Bayesian modeling. Instead, we use the weaker assumption called \qu{exchangeability}. Look in the book or on Google and give the definition for exchangeability.}\spc{3}


\easysubproblem{[MA] When we collect data for an experiment are the observations exchangeable? Is this an assumption we make use of all the time?}\spc{2}

\easysubproblem{[MA] Look up de Finetti's respresentation theorem. State it here.}\spc{6}

\easysubproblem{[MA] Does exchangeability imply conditional independence for some distribution $\prob{X~|~\theta}$? Yes / No.}\spc{3}
\end{enumerate}